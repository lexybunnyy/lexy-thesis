	A felület manuális teszteléssel lett kipróbálva, automatizált tesztesetek nem lesznek.
\subsection{Kalkulátor tesztelés}
	Ebben a részben a cpp-ben írt tesztesetekről lesz szó. \newline
	A tesztelést folyamatosan végeztem a minta adatok alapján. A függvények implementálása közben ezekre a minta adatokra meghívtam, majd ezekkel számoltam. A teszteléshez a logTest.cpp fájlban található függvényeket alkalmaztam. \newline
	A fájlban a javítást segítő kiírató függvények, integrációs teszt esetek, valamint manuális, felület nélküli számítást végző tesztesetek vannak. \newline
	Ezeket a teszteket nagyrészt kiváltották a szerveren futó tesztek, így fejlesztésük abba maradt, előfordulhat hogy a tesztek elavultak.
	\begin{description}
		\item[bool testAll()] \hfill \\ 
			Minden teszt lefuttatása. Ha minden teszt jól futott le akkor "true" értékkel tér vissza.
		\item[void testInterpolation(bool logPoly = false)] \hfill \\ 
			Interpolációs teszteket futtat.
			Mindegyik teszt egy egyszerű interpolációt vesz alapul, és a számítás eredményét ellenőrzi.
		\item[bool testMainInterpolation(bool logPoly = false)] \hfill \\ 
			Fő függvény tesztje. Elsősorban a feltétel átmeneteket teszteli, és azt hogy valóban számol-e ezen keresztül interpolációt.
		\item[bool testNewton(bool logPoly = false)] \hfill \\ 
			Newton számítás tesztje.
		\item[bool testLagrange(bool logPoly = false)] \hfill \\ 
			Lagrange interpoláció tesztje.
		\item[void testPolynomial()] \hfill \\ 
			Interpolációs mátrix tesztje.
		\item[void testMatrixInterpolation()] \hfill \\ 
			Manuális mátrix számítás függvénye.
		\item[void testManualInterpolation()] \hfill \\ 
			Manuális interpoláció számítás függvénye.

		\item[testManualPolynomial()] \hfill \\ 
			Manuális polinom összeadás és szorzás függvénye. 
		\item[void genXSquaredPoints(DArray \&X, DMatrix \&Y)] \hfill \\ 
			generál egy minta X,Y ponthalmazt az $x^{2}$ 
			pontjaiból.
		testMatrixInterpolation Segédfüggvénye
		feltölti az $x^{2}$ pontjaival.
	\end{description}
\subsection{Komponens és integrációs tesztelés}
	Főként az elosztást és a párhuzamosítást, valamint a szerveren lévő adatfeldolgozást teszteltem ilyen formában. Ezekkel a tesztesetekkel ellenőriztem a szerver és a számítás helyességét, és a szerver futást, miközben fejlesztettem.\newline
	Ezeket a teszteket érdemes lefuttatni, amikor a szervert konfiguráljuk. A szerveren lévő komponensek és azok egymással való kommunikációja fut le ilyenkor. Ha a teszten átmennek, akkor nagy valószínűséggel a szerveren már nem lehet probléma.\newline
	Viszont ha egy modul rosszul van leforgatva, vagy nincs betöltve, a tesztek hibát jeleznek. Ha a szerveren vagy gépen nem sikerült a tesztek lefutása, nem lehetséges a számítás elvégzése. Ennek oka valamelyik modul rosszul való betöltése, vagy az Erlang, GCC verziója nem megfelelő a számításokhoz.\newline
	Az átfogó tesztelés a ServerConfig/test.erl fájlban található. 
		\begin{description}
		\item[test:fork] \hfill \\
			Teszt futtatása a fork-nak ez a teszteset az párhuzamosítás miatt volt fontos. Viszont egy másik teszt átvette a helyét.
			\newline
			Ha mégis szükség lenne az általános párhuzamosítás tesztelésére, ezt kellene továbbfejleszteni.

		\item[test:runCheck, run] \hfill \\
			Futtatást kezelő függvények.
			Lefuttatják azokat a teszteket amiket a \\ 
			\texttt{ServerConfig/main.erl}-ből már el lehet indítani. 
			\newline \texttt{test:simulateDistributedCalculate} függvényt a main-ben található függvény tesztelésére használjuk, amit a bin/run.erl-ben lévő teszt futtat.
		\item[test:simulateDistributedCalculate] \hfill \\
			Egy olyan minta adatból, melyet a szerver is küldhet, elvégzi a számítást és ellenőrzi az eredményt. \newline
			Ha megadjuk neki a node-figyelő pid-jét akkor elosztott számítást is teszteli, és a node-figyelővel való kommunikációt. 
		\item[test:simplifyPolinomialTest] \hfill \\
			Ellenőrzi a struktúra kezelőben megvalósított polinom egyszerűsítést.  
		\item[test:convertMochiElements] \hfill \\
			structHandler: getTableData és getElementByKey tesztelésére írt függvény.
		\item[test:getResultTest] \hfill \\
			Eredmény konvertálásának tesztje. Szimulál egy processzektől visszakapható eredményt, majd ezt átalakítja a küldéshez megfelelő formátummá.
		\item[test:getParseJSONParams] \hfill \\
			Json string-ből a számításhoz szükséges minden paraméter kinyerésének tesztelése (structHandler teszt).
		\item[test:convertStruct] \hfill \\
			structHandler függvényeinek tesztje: \newline
			getNewPointStruct, appendNewPointStruct, convertPoints.
		\item[test:simulateFirstParseAndRun] \hfill \\
			Kibontja az első elemet a mintából, és számítás után ellenőrzi az eredmény helyességét. \newline
			Ez a teszt a számítást és az adat konvertálást teszteli, a párhuzamosítást/elosztást nem.
		\item[test:getFirstElementOfDataSet] \hfill \\
			Visszaadja a minta adatok első elemét.
		\item[test:getJSONString] \hfill \\
			Egy minta adathalmazt ad vissza (json string) ami jöhet a felületről.
		\item[test:getResultTestHelper] \hfill \\
			Számítási eredményekből és az elvárt eredményből a tesztelés eredményt állítja elő. 
	\end{description}

\subsection{Manuális tesztelés}
	Elsősorban a felület átfogó tesztelését végeztem ilyen módon, de a szerverrel való kommunikációnál és az elosztásnál is előkerültek ilyen módon hibák. \newline
	Az alábbi táblázatban felsoroltam a hibákat, melyek feltűntek a tesztelés során. 
	\begin{center}
  	\begin{tabular}{| p{4cm} | p{1.5cm} | p{8cm} |}
    \hline
    Hiba & Javítva & Info
  	\\ \hline
        Nem jó pontosság esetén nem jelenik meg a polinom
      &
      	igen
      &
		\textbf{Előidézés} : beírsz betűket a pontossághoz 
    \\ \hline
        Egy interpoláció adatbetöltési hibák
      &
      	igen
      &
      	Egy interpoláció tulajdonságainak betöltésénél nem állítja be az interpoláció típust és azt hogy inverz-e.
      	Nem jelenítjük meg a polinomokat sem bármilyen egyéb formában.
    \\ \hline
        Eredmény betöltésnél előfordul hogy a pontosság undefined 
      &
      	igen
      &
		\textbf{Előidézés}:  Minta adatban fordul csak elő, de ha onnan  is betöltődhet akkor máshonnan is, amikor betölti az adatokat le kellene ezt kezelni.
   \\ \hline
    	Amikor pontot adunk hozzá nem frissül 
      &
      	igen
      &
		\textbf{Előidézés}: Pont hozzáadása gomb esetén nem mentődnek el a háttérben az adatok
    \\ \hline
    	Hermite inverz nincs implementálva, és mégis beállítható a felületen.  
      &
      	igen
      &
		El tudjuk küldeni az oldalról úgy az adatokat hogy Hermite interpolációnál is állítható az inverz, közben a szerver nem inverz interpolációt fog számolni.
	\\ \hline
    	Listában a törlés gomb nem működik megfelelően.    
      &
      	nem
      &
		Interpolációk listájában a törlés gomb hatására nem törli ki az adott sort. 
	\\ \hline
  \end{tabular}\end{center}
  \begin{center}\begin{tabular}{| p{4cm} | p{1.5cm} | p{8cm} |}
  \hline Hiba & Javítva & Info
    \\ \hline
      	Apache/2.2.22 verziónál a szerver nem működik. 
      &
      	igen
      &
		\texttt{[Sat May 09 17:20:18 2015] [crit] [client 192.168.1.153] configuration error:  couldn't perform authentication. AuthType not set!: /}\newline
		Server version: Apache/2.2.22 (Debian)
		Server built:   Jan 10 2015 15:33:51 
		\newline\textbf{Megoldás}:
		a régebbi verzió nem tudja kezelni a \texttt{"Require all granted"} kulcsszót.
	\\ \hline
        Node Lista lekérdezésnél valamikor végtelen ciklusba fut a lekérdezés. 
      &
      	igen
      &
      	\textbf{Hiba}:
		Hiba: túl sokszor küldjük el ugyanazt az üzenetet, és kapunk vissza rossz választ, kéne bele egy időkorlát is, a várakozásra. Másik hiba pedig hogy így egy régebbi üzenet lekezelése is megtörténhetett, így nem a legfrissebb node listát kaptuk meg. 
		\newline 
		\textbf{Megoldás}: Csak egyszer küldjük el, és ha rossz válasz érkezik, akkor tovább figyelünk.
  	\\ \hline
    	Szerver hiba bizonyos mennyiségű adatküldése felett 
      &
    	igen
      &  
    	\textbf{Előidézés}: 9-nél több egyszerű adat felküldése, minta nagy adatból 4db felett.
    	A szerver egybe küldi az értékeket, a httpServer modul már nem kapja meg a végét. \newline
    	\textbf{Hiba}: A tcp kapcsolat nem záródik le, ezért bele kellett nyúlni az eredeti minta kód logikájába is.\newline
    	\textbf{Megoldás}: Bizonyos idő után lezárjuk a kapcsolatot Timeout-al, olyan idő lett megadva amelyben már remélhetőleg minden csomag megérkezik.
    \\ \hline
	\end{tabular}\end{center}

  	\begin{center}\begin{tabular}{| p{4cm} | p{1.5cm} | p{8cm} |}
	\hline Hiba & Javítva & Info 
    \\ \hline
        Ha kilép egy node, akkor elszáll a számítás 
      &
      	nem
      &
      \textbf{Ötletek}:
      Node-ok kivételére is kellene opció,vagy kezelni kellene az elszálló node-okat.
      \newline Mielőtt elkezdjük a számítást, mindegyik node-ot ellenőrizzük, amelyik nem létezik már azt kivesszük a listából. Ha számítás közben hal meg: Bizonyos időközönként szintén küldünk egy ping-et az adott szervernek hogy létezik-e még. Ha meghalt a node akkor hibával tér vissza az adott számítás a kliensnek.
   	\\ \hline
    	Erlang: elosztási hiba
      &
      	nem
      &
      	Előfordul hogy a szerver elküldi a számító node-nak az adatot, az ki is számolja, viszont az eredmény nem érkezik vissza.\newline
      	A hiba elég véletlenszerűnek tűnik. Többször is elő tudtam igézni, majd amikor létrehoztam teljesen új elosztott rendszert egy vagy két gépen akkor már nem jelentkezett a hiba. 
    \\ \hline
    	Erlang run:teszt és telepítés hiba a szerveren erlang 5.9.1 es verziónál
      &
      	nem
      &
      	\textbf{Telepítő futtatási hiba}: \newline
      	\texttt{escript: Internal error: badarg}
		\texttt{"init terminating in do\_boot"}
        \texttt{\{badarg,[\{io,format,[<0.29.0>,"~p",}
        \texttt{[[\{io\_lib,format ... }
		\newline \textbf{Teszt futtatásánál hiba}: \newline
		\texttt{exception error: no case clause matching <<127,248,0,0,0,0,0,0>>}
     	\texttt{in function  io\_lib\_format:mantissa\_exponent\/1}
     	\texttt{(io\_lib\_format.erl, line 374) }
    \\ \hline
  	\end{tabular}\end{center}