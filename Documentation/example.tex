\documentclass{elteikthesis}

\usepackage{ucs}
\usepackage[utf8x]{inputenc}
\usepackage[T1]{fontenc}
\usepackage[english,hungarian]{babel}
\selectlanguage{hungarian}

\usepackage{listings}
\usepackage{color}

\definecolor{mygray}{rgb}{0.4,0.4,0.4}
\definecolor{mygreen}{rgb}{0,0.8,0.6}
\definecolor{myorange}{rgb}{1.0,0.4,0}

\lstset{
basicstyle=\footnotesize\sffamily\color{black},
commentstyle=\color{mygray},
frame=single,
numbers=left,
numbersep=5pt,
numberstyle=\tiny\color{mygray},
keywordstyle=\color{mygreen},
showspaces=false,
showstringspaces=false,
stringstyle=\color{myorange},
tabsize=2
}

\title{Interpoláció osztott rendszereken}
\author{Cselyuszka Alexandra}
\supervisor{Tejfel Máté}
\supervisorstitle{egyetemi tanár}
\period{Informatika Bsc}
\thesisyear{2015}
\department{Programozási Nyelvek és Fordítóprogramok Tanszék}
\additionaltext{ABCDEF GHIJKLM NOPQRSTUV WXYZ}

\begin{document}

\frontmatter

	\maketitle
	\tableofcontents
	
\mainmatter

\chapter{Bevezetés} 
a témaválasztás indoklását és a megoldandó feladat rövid, közérthető leírását
tartalmazza.

\chapter{Felhasználói dokumentáció}
A felhasználó megnyitja a weboldalt, megtekintheti a táblázatokat és a grafikonokat. Minden adatot szerkeszthet és új adathalmazt hozhat létre.

\section{Weboldal}
Weboldalon kattingat majd szép eredményeket kap, pontokat lát aztán ha kiszámolja az ereményt még szebb polinómot kap eredményül ha sikeres volt a számítás.

\chapter{Fejlesztői dokumentáció}
A program 3 fő részből áll a Weboldalból, az Elosztott rendszerből és a Kalkulátorból.

\section{Weboldal}
Weboldal felépítése HTML és JavaScript segítségével valósult meg. Egy oldalból áll melyen a felhasználó össze állítja a neki szükséges adathalmazt. Új adathalmazokat hozhat létre, a régieket szerkesztheti. A háttérben JSON-be formálódnak az adatok, melyeket a felhasznéló is láthat, ha debug-módban lép be. 
Ha a felhasználó végzett egy gombra nyomással a program legenerálja a szükséges JSON-t. 

\section{Elosztott rendszer}
Elosztott rendszer Erlang-ban lett megvalósítva. Az elosztást Interpolációnként végezzük, vagyis annyi node-ot hozunk létre amennyi Interpolációt kívánunk egyszerre kiszámítani.
\subsection{Adat feldolgozás}
Az elosztott rendszer először kap egy JSON adathalmazt melyből kinyeri a neki szükséges adatokat, és átkonvertálja.

\section{Kalkulátor}
A Kalkulátor részben számítódik ki egy-egy Interpolációnak az ereménye.
A megkapott adatok alapján számol, ha kell létre hozza a kezdő mátrixot, kiszámolja az eredmény mátrixot, majd annak segítségével kiszámolja a polinómot.
\begin{lstlisting}[language=C++]{Name=test2}
	DArray interpolateMain (
		DArray &x, DMatrix &Y, 
				string type = "lagrange", bool inverse = false);
\end{lstlisting}
	Kívülről meghívandó fő függvény mely elosztja és konvertálja a részeket. 
	\begin{description}
	  \item[DArray \&x] \hfill \\ Az x pontok listája 
	  \item[DMatrix \&Y] \hfill \\ Az x pontokhoz tartozó y pontok halmaza
	  \item[string type] \hfill \\ Interpóláció típusa: lagrange, newton, hermite
	  \item[bool inverse] \hfill \\ Inverz Interpoláció kell-e
	\end{description}


\section{Kommunikáció}
A 3 különállóan megvalósított program részlet speciális módon kommunikál egymással. 
\subsection{Kalkulátor és az Elosztott rendszer közötti komunikáció} 
Az elosztott rendszerben hívódó számítást Erlang - erl\_nif"-el sikerült megoldanom. 
Az ezzel kapcsolatos dolgokat az Calculator/erlang.cpp tartalmazza. 


\chapter{Források}
\begin{itemize}
\item {http://www.erlang.org/doc/man/erl\_nif.html}
\item {https://www.sharelatex.com/learn/Sections\_and\_chapters}
\item {https://github.com/mochi/mochiweb/blob/master/src/mochijson.erl} 
\item {http://tex.stackexchange.com/questions/137055/lstlisting-syntax-highlighting-for-c-like-in-editor} 
\end{itemize}

\end{document}
