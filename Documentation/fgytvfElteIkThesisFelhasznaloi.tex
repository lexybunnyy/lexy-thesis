\begin{comment}
	A Felhasználói dokumentáció tartalmazza
	- a megoldott probléma rövid megfogalmazását,
	- a felhasznált módszerek rövid leírását,
	- a program használatához szükséges összes információt

	Magába foglalja a telepítési- (vagy üzemeltetési-) és a végfelhasználói leírást. Ezek
	meghatározott célközönséghez szólnak, könnyen és gyorsan kell, hogy eligazítsák a
	felhasználót a program használatában!

\end{comment}

\section{Bevezetés}
%% A feladat rövid ismertetése (mire való a szoftver)
%% Célközönség (kik, mikor, mire használhatják a programot)
\section{Telepítési útmutató}
\subsection{Rendszer követelmények}
%% A rendszer használatához szükséges minimális, illetve optimális HW/SW környezet
	A szoftver Linux Mint 17.1 'Rebecca' Cinnamon 64-bit operációs rendszeren lett tesztelve. 
	\begin{description}
		\item[Operációs rendszer:] debian alapú rendszer (debian, ubuntu, mint)
		\item[Apache:] 2.4.7 (Ubuntu)
		\item[Erlang:] 5.10.4 (SMP,ASYNC\_THREADS) (BEAM) emulator 
		\item[g++:] 4.8.2 (Ubuntu)
		\item[Mozilla Firefox:] 37.0.2
	\end{description}
	\begin{verbatim}
	sudo apt-get install gcc-4.8
	sudo apt-get install g++-4.8
	sudo update-alternatives --install /usr/bin/g++ g++ /usr/bin/g++-4.8 20
	sudo apt-get install erlang
	sudo apt-get install apache2
	sudo apt-get install libapache2-mod-proxy-html
	sudo apt-get install libxml2-dev
	\end{verbatim}

%% Első üzembe helyezés leírása – ha van ilyen –, a program indítása (kivéve, ha nem egy
%%önálló alkalmazásról, hanem egy meglévő rendszer új komponenséről van szó). Itt
%%ellenőrizzük, hogy a telepítési útmutató megfelel-e a valóságos telepítési folyamatnak.
\subsection{Szerver és segédgépek üzembe helyezése}
	\begin{verbatim}
		git clone git+ssh://oem@192.168.1.153/~/git/project
		sudo cp ./../ServerConfig/szakdoli.conf /etc/apache2/sites-available/szakdoli.conf

		sudo vim /etc/apache2/sites-available/szakdoli.conf
		DocumentRoot /media/usbhdd/data/lexy/project/WebPage/
        <Directory /media/usbhdd/data/lexy/project/WebPage/>
            Options FollowSymLinks

        sudo a2enmod proxy proxy_http
        sudo a2ensite szakdoli.conf
		sudo /etc/init.d/apache2 restart
	\end{verbatim}

\subsection{Használati útmutató}
\subsubsection{Weboldal}
\subsubsection{Szerver}
%% Általános felhasználói tájékoztató (például a szokásostól eltérő képernyő-, billentyű-,
%%illetve egérkezelés leírása, teendők hibaüzenetek esetén stb.).
%% A rendszer funkcióinak ismertetése. A feladat jellegéből fakadóan célszerű lehet ezt
%%folyamatszerűen, képernyőképekkel alátámasztva bemutatni. A funkciókat ajánlatos a
%%felhasználói szintek szerint csoportosítani. Itt vegyük figyelembe, hogy a leírás a
%%fejlesztői dokumentációban meghatározott részfeladathoz illeszkedik-e, az ott
%% meghatározott funkciókat/használati eseteket írja-e le?
%% A rendszer futás közbeni üzenetei (hibaüzenetek, figyelmeztető üzenetek, felszólító üze-
%%netek stb.) és azok magyarázata – az esetleges üzemeltetési teendőkkel együtt. Itt vegyük
%%figyelembe, hogy tartalmaz-e biztonsági, illetve hibaelhárítási előírásokat?
%% Egyéb, a szoftver használatához szükséges információk.